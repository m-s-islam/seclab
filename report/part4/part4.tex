\documentclass[12pt]{report}
\usepackage{fancyhdr}
\usepackage[a4paper]{geometry}
\usepackage[myheadings]{fullpage}
\usepackage{lastpage}
\usepackage{graphicx, wrapfig, subcaption, setspace, booktabs}
\usepackage[T1]{fontenc}
\usepackage[font=small, labelfont=bf]{caption}
\usepackage{fourier}
\usepackage[protrusion=true, expansion=true]{microtype}
\usepackage[english]{babel}
\usepackage{sectsty}
\usepackage{url, lipsum}
\usepackage{float}
\usepackage{listings}
\usepackage[T1]{fontenc}
\usepackage{xcolor}
\usepackage{lmodern}
\usepackage{listing}

\lstset{escapeinside={<@}{@>}}

\newcommand{\HRule}[1]{\rule{\linewidth}{#1}}
\onehalfspacing
\setcounter{tocdepth}{5}
\setcounter{secnumdepth}{5}


% header and footer

\pagestyle{fancy}
\fancyhf{}
\setlength\headheight{15pt}
\fancyhead[L]{Part 4: Implementing Secure Web Applications}
\fancyhead[R]{Security Insider Lab II}
\fancyfoot[R]{Page \thepage\ of \pageref{LastPage}}

% Title page

\begin{document}
	
	\title{ \normalsize \textsc{Lab Report}
		\\ [2.0cm]
		\HRule{0.5pt} \\
		\LARGE \textbf{\uppercase{Security Insider Lab II \\
				Part 4: Implementing Secure Web Applications}}
		\HRule{2pt} \\ [0.5cm]
		\normalsize \vspace*{4\baselineskip}
		\LARGE {\tt Group 5}\\}
	\date{}
	\author{
		 \\
		Abhijeet Patil \\
		Mohammad Saiful Islam\\
		Thejeswi Preetham Nagendra Kamatchi}
	\maketitle
	\newpage
	
	
	\section*{Exercise 1: White-Box Web Application Vulnerability Testing}
	\newpage
	
	\section*{Exercise 2: Black-Box Web Application Vulnerability Testing}
	
	\paragraph*{1.}{\bf Download two (or more) web vulnerability scanners and describe how you setup all the appropriate environment settings needed.}\\
	
	For Black-box web application vulnerability testing, we have downloaded OWASP Zed Attack Proxy(also known as OWASP ZAP), nikto and uniscan.\\
	
	{\bf Nikto:} Nikto is a small and simple tool examines a website and reports back to you the potential vulnerabilities that it found that you could use to exploit. 
	Using Nikto is very straight-forward, as the following command:\\\\
	{\tt nikto -h [hostname or ip]}
	\\or\\
	{\tt perl nikto -host [hostname or ip]}
	\\\\Our vBank web application is located at {\sf http://192.168.0.29/vBank}. Here is the screenshot of the nikto scan:
	
	\begin{figure}[H]
		\includegraphics[width=0.75\textheight]{images/nikto.jpg}
		\caption{request a loan}
	\end{figure}
	
	

\end{document}